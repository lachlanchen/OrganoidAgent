\documentclass[aspectratio=169]{beamer}

\usetheme{Boadilla}
\usecolortheme{seahorse}
\usepackage[T1]{fontenc}
\usepackage[utf8]{inputenc}
\usepackage{graphicx}
\usepackage{hyperref}

\title{Agentic Organoid Analysis for Immune Migration and Growth Forecasting}
\subtitle{A Nature Methods-level plan for AI-driven organoid image analysis}
\author{OrganoidAgent Project}
\date{\today}

\begin{document}

\begin{frame}
  \titlepage
\end{frame}

\begin{frame}{Motivation}
  \begin{itemize}
    \item Organoid experiments create large brightfield datasets with limited quantitative analysis.
    \item Immune cell migration and organoid growth are dynamic and require robust tracking over time.
    \item We need an end-to-end pipeline that is reproducible, quality-aware, and scalable.
    \item Goal: turn raw images into validated phenotypes and predictive models.
  \end{itemize}
\end{frame}

\begin{frame}{Core Scientific Questions}
  \begin{itemize}
    \item How do immune cells recruit to and infiltrate organoids under cytokine stimulation?
    \item Which perturbations alter motility modes and organoid damage phenotypes?
    \item Can early timepoints forecast later organoid growth and reduce assay duration?
  \end{itemize}
\end{frame}

\begin{frame}{Datasets in Scope}
  \begin{itemize}
    \item PBMC + organoid co-culture images (brightfield TIFFs, multiple conditions and days).
    \item Science Advances paper PDF: MAIT cell activation and recruitment (sciadv.adn6331).
    \item Organoid time-course dataset (days D0--D14; multiple passages and experiments).
    \item Optional: supplementary PPTX decks with experimental notes and labels.
  \end{itemize}
\end{frame}

\begin{frame}{Dataset Inventory (PBMC + Organoid)}
  \begin{itemize}
    \item 4,893 image files (TIFF/JPEG/PNG).
    \item Condition tags in filenames: IL12+18, IL1b+23, IL2+7, Unstim, Matrigel, Organoid, Organoid+EC.
    \item Timepoints present: D0--D3 dominant, with sparse D4--D10.
    \item Multiple experiment groups (App61+P42, App61+P62, App61+P72, etc.).
  \end{itemize}
\end{frame}

\begin{frame}{Dataset Inventory (Organoid Time Course)}
  \begin{itemize}
    \item 2,824 image files across multiple experiments and passages.
    \item Timepoints: D0, D1, D2, D3, D4, D5, D6, D7, D8, D9, D10, D11, D12--D14.
    \item Passages: P3--P12 with multiple dates and replicates.
  \end{itemize}
\end{frame}

\begin{frame}{Why Brightfield (Method Justification)}
  \begin{itemize}
    \item Label-free imaging avoids phototoxicity and label bias in immune motility.
    \item Stable over long time scales (hours to days) for migration and growth trends.
    \item Captures mechanical phenotypes: deformation, boundary interactions, collective behavior.
  \end{itemize}
\end{frame}

\begin{frame}{Proposed System: Agentic Analysis Pipeline}
  \begin{enumerate}
    \item Data intake and metadata normalization.
    \item Image quality control (focus, illumination, artifact detection).
    \item Organoid segmentation and boundary tracking.
    \item Immune cell detection and tracking.
    \item Migration phenotyping and organoid damage metrics.
    \item Growth forecasting from early timepoints.
    \item Report agent for summary, QC, and interpretable results.
  \end{enumerate}
\end{frame}

\begin{frame}{Agent Architecture (Practical, Non-Sci-Fi)}
  \begin{itemize}
    \item Agent = workflow controller + self-checker, not a black-box scientist.
    \item Each module produces artifacts + confidence and can be re-run on failure.
    \item LLM used for structured reporting and decision summaries only.
  \end{itemize}
\end{frame}

\begin{frame}{Module 1: Quality Control}
  \begin{itemize}
    \item Focus metrics (Laplacian variance, Tenengrad).
    \item Illumination stability and background variance.
    \item Bubble/debris detection using simple heuristics or lightweight classifier.
    \item Output: QC score per frame and per well with failure reasons.
  \end{itemize}
\end{frame}

\begin{frame}{Module 2: Organoid Segmentation}
  \begin{itemize}
    \item Candidate models: Cellpose, SAM/SAM2, classical threshold + morphology.
    \item Outputs: mask, boundary contour, size/shape descriptors, confidence.
    \item Temporal stability check across frames.
  \end{itemize}
\end{frame}

\begin{frame}{Module 3: Immune Cell Detection and Tracking}
  \begin{itemize}
    \item Detection-first: background subtraction + blob detection.
    \item Optical-flow-first: detect moving objects when contrast is low.
    \item Tracking: Kalman + Hungarian or TrackMate-style association.
    \item Outputs: tracks, ID-switch statistics, track confidence.
  \end{itemize}
\end{frame}

\begin{frame}{Module 4: Migration Phenotyping}
  \begin{itemize}
    \item Flux into a boundary ring and dwell time.
    \item Chemotaxis index and radial velocity toward organoid center.
    \item Infiltration score (boundary crossings with persistence).
    \item Condition-level distributions with bootstrap confidence intervals.
  \end{itemize}
\end{frame}

\begin{frame}{Module 5: Growth Forecasting}
  \begin{itemize}
    \item Predict later-day morphology from early-day images (e.g., D0--D3 to D7--D10).
    \item Targets: size distribution, growth rate, failure risk.
    \item Modeling options: temporal regression, sequence models, or growth curve fitting.
  \end{itemize}
\end{frame}

\begin{frame}{Module 6: Report Agent}
  \begin{itemize}
    \item Summarize QC, segmentation confidence, tracking metrics.
    \item Provide condition-level phenotypes and outliers.
    \item Generate reproducible, structured reports for lab and paper.
  \end{itemize}
\end{frame}

\begin{frame}{Validation Plan (Nature Methods Bar)}
  \begin{itemize}
    \item Organoid boundary accuracy: Dice/IoU and boundary error.
    \item Immune cell tracking: IDF1, MOTA, ID switches on annotated clips.
    \item Biological validation: known cytokine effects on recruitment and infiltration.
    \item Cross-batch generalization: hold-out experiments and passages.
  \end{itemize}
\end{frame}

\begin{frame}{Baselines and Comparators}
  \begin{itemize}
    \item Classical thresholding and morphology-only pipelines.
    \item Single-module baselines (segmentation without QC).
    \item Human manual quantification on small subsets.
  \end{itemize}
\end{frame}

\begin{frame}{Reproducibility and Release}
  \begin{itemize}
    \item Publish dataset manifest and evaluation subset.
    \item Provide scripts for QC, segmentation, tracking, phenotyping.
    \item Release reports with standardized metrics and plots.
  \end{itemize}
\end{frame}

\begin{frame}{Timeline (12 Months)}
  \begin{itemize}
    \item Months 1--2: QC module, metadata manifest, baseline segmentation.
    \item Months 3--5: immune cell detection + tracking + initial phenotypes.
    \item Months 6--8: growth forecasting and cross-condition comparisons.
    \item Months 9--10: benchmarking, ablation, and validation.
    \item Months 11--12: manuscript figures, reproducibility package, release.
  \end{itemize}
\end{frame}

\begin{frame}{Risks and Mitigations}
  \begin{itemize}
    \item Low contrast immune cells: add motion-based detection and temporal smoothing.
    \item Artifact-heavy frames: QC gating with explicit failure reasons.
    \item Heterogeneous naming: normalize with a metadata manifest and regex rules.
  \end{itemize}
\end{frame}

\begin{frame}{Expected Figures for Paper}
  \begin{itemize}
    \item Pipeline schematic and agent architecture.
    \item QC acceptance examples and artifact taxonomy.
    \item Segmentation + tracking overlays.
    \item Migration phenotype comparisons (boxplots, CIs).
    \item Growth forecasting curves and error metrics.
  \end{itemize}
\end{frame}

\begin{frame}{Claimed Contributions (Nature Methods)}
  \begin{itemize}
    \item Quality-aware, end-to-end pipeline for brightfield immune migration analysis.
    \item Unified phenotyping of recruitment, infiltration, and damage metrics.
    \item Growth forecasting to reduce assay time.
    \item Reproducible benchmarks and reporting artifacts.
  \end{itemize}
\end{frame}

\begin{frame}{Next Steps}
  \begin{itemize}
    \item Generate standardized manifest from filenames and folders.
    \item Label a small benchmark subset for segmentation and tracking validation.
    \item Produce initial QC and migration metrics for IL12+18 vs Unstim.
  \end{itemize}
\end{frame}

\begin{frame}{Thank You}
  \begin{center}
    Questions and feedback welcome.
  \end{center}
\end{frame}

\end{document}
